\documentclass{article}
\usepackage{graphicx} % Required for inserting images

\usepackage[a4paper, total={7in, 10.5in}]{geometry}
\usepackage{hyperref}

\usepackage{amsmath}
\usepackage{amssymb}
\usepackage{amsthm}

\newtheorem{theorem}{Theorem}%[section]
\newtheorem{lemma}[theorem]{Lemma}
\newtheorem{corollary}[theorem]{Corollary}


\theoremstyle{definition}
\newtheorem{definition}[theorem]{Definition}
\newtheorem{example}[theorem]{Example}
\newtheorem{assumption}[theorem]{Assumption}


\AtBeginEnvironment{definition}{\renewcommand\em{\bfseries\textit}}
\renewcommand\em{\bfseries}

\newcommand{\agap}{\hspace{0.25em}}
\newcommand{\dpair}[2]{\left\langle #1, #2 \right\rangle}
\newcommand{\subterm}{\sqsubset}
\newcommand{\subtermeq}{\sqsubseteq}
\newcommand{\supterm}{\sqsupset}
\newcommand{\suptermeq}{\sqsupseteq}


\usepackage[backend=biber]{biblatex}
\addbibresource{./ref.bib}

\title{The Relative Chain Framework for Modular Termination}
\author{\href{mailto:oi24939@bristol.ac.uk}{\texttt{Tom.Divers@bristol.ac.uk}}, \href{mailto:eddie.jones@bristol.ac.uk}{\texttt{Eddie.Jones@bristol.ac.uk}}}
\date{}
 
\begin{document}

\maketitle

\section{Introduction}

We present a new framework of \emph{relative chains} for proving the termination of the union of two terminating rewrite systems. We demonstrate how our framework can be used to prove the termination of functional programs augmented with equational hypotheses, and \emph{...}

\section{Preliminaries}

In this section, we discuss some mathematical preliminaries concerning term rewriting and reduction relations. The foundations of this field are explored in depth in \cite{baader1998terms}.

\subsection{Term Rewrite Systems}

We consider the setting of term rewriting systems over a \emph{signature} $T(\Sigma, V)$. A signature consists of a finite set of \emph{function symbols} $\Sigma$ and an infinite set of \emph{variables} $V$. $\Sigma$ is presumed to be partitioned into a set of \emph{defined functions} $\Sigma_{def}$ and \emph{constructors} $\Sigma_{con}$. Each $f \in \Sigma$ has an associated natural number called its \emph{arity}, and $\Sigma^{(i)}$ denotes all the function symbols in $\Sigma$ with arity $i$.

$T(\Sigma, V)$ is defined inductively as follows: \begin{enumerate}
    \item $\Sigma^{(0)} \subseteq T(\Sigma, V)$ and $V \subseteq T(\Sigma, V)$. 
    \item If $f \in \Sigma^{(n)}$, and $M_i \in T(\Sigma, V)$ for each $i \in [n]$, then $f(M_1, \cdots, M_n) \in T(\Sigma, V)$. 
\end{enumerate}

The recursive structure of $T(\Sigma, V)$ prompts us to think of terms in a TRS as \emph{syntax trees}. We define $Pos(M) \subseteq \mathbb{N}^*$ for each $M \in T(\Sigma, V)$ to be the set of positions in $M$'s syntax tree. The root of the tree has position $\varepsilon$, and the $i$th child of a node at position $p$ has position $pi$. $M|_p$ is the subterm of $M$ rooted at position $p$, and $M[N]_p$ denotes $M$ with its term at position $p$ replaced by $N \in T(\Sigma, V)$. Additionally, we define $Var(M) \subseteq V$ to be the set of variables appearing in a term. $M$ is \emph{closed} iff $Var(M) = \emptyset$. 

A \emph{(one-hole) context} $C[\cdot] : T(\Sigma, V) \rightarrow T(\Sigma, V)$ is a term with a hole $\square$ at some position. $C[M]$ denotes the context $C[\cdot]$ with its hole replaced by the term $M$. 

A \emph{substitution} $\theta : X \rightarrow T(\Sigma, V)$ is a function mapping variables to terms, for which the set $\text{dom}(\theta) := \{ x \in V ~|~ \theta(x) \neq x\}$ is finite. Hence, we may write $\theta = [x_1 \mapsto M_1, x_2 \mapsto M_2, \cdots]$ for finitely many $x_i \in V$ and $M_i \in T(\Sigma,V)$. $M \theta$ denotes the term $M$ with each variable replaced by its $\theta$-image, and we call $M\theta$ an \emph{instance} of $M$. For two substitutions $\theta, \sigma$, we say that $\sigma$ is \emph{less general} than $\theta$ (and write $\sigma \leq \theta$) iff there exists some other substitution $\theta'$ such that $\sigma = \theta' \circ \theta$.
\\~\\
A reduction system $(X, \rightarrow)$ consists of a set $X$ equipped with a binary relation $\rightarrow ~\subseteq X \times X$. A reduction is \emph{terminating} iff there exist no infinite sequences $x_1x_2 \cdots \in X^\omega$ with $x_1 \rightarrow x_2 \rightarrow \cdots$. $x \in X$ is a \emph{redux} of $\rightarrow$ iff $x \rightarrow y$ for some $y \in X$. If $x$ is not a redux, we say that it is in \emph{$\rightarrow$-normal form}. A reduction system over $T(\Sigma, V)$ is a \emph{term rewrite system (TRS)} iff it is closed under contexts and substitutions (i.e. if $M \rightarrow N$, then $C[M\theta] \rightarrow C[N\theta]$ for all contexts $C[\cdot]$ and substitutions $\theta$). 

Here, we will consider TRSs defined by a set of equations $R \subseteq T(\Sigma, V) \times T(\Sigma, V)$. Note that our equations are presumed to be \emph{oriented}, meaning that $M \approx N \in R$ does not imply that $N \approx M \in R$. We also assume that $Var(M) \supseteq Var(N)$ for each $M \approx N \in R$. We define $\rightarrow_R$ such that, for each $M \approx N \in R$, and for all contexts $C[\cdot]$ and substitutions $\theta$, $C[M\theta] \rightarrow_R C[N\theta]$. 

An equation $M \approx N$ is \emph{stable} iff $M$ is headed by a defined function symbol. A TRS is stable iff all of its equations are stable. A TRS is a \emph{functional program} iff it is stable, and for each rule $f(x_1, \cdots, x_n) \approx N$, each $x_i$ contains no defined function symbols. We say that an equation $M \approx N$ is \emph{$Q$-normal} (where $Q$ is some TRS) iff $N$ is in $Q$-normal form. A TRS is $Q$-normal iff all of its rules are $Q$-normal.  
\\~\\
A strict \emph{reduction order} $\succ ~ \subseteq T(\Sigma, V)^2$ is an order on terms that is closed under substitutions.  $\succsim$ is defined to be the reflexive closure of $\succ$. A reduction order is also \emph{monotonic} iff it is closed under contexts. $\succ$ is \emph{well-founded} iff every set $K \subseteq T(\Sigma, V)$ has a minimum under $\succ$ (i.e. $\forall K \subseteq T(\Sigma, V), \exists x \in K, \forall y \in K, x \precsim y$).

One particular ordering that will be of use is the subterm ordering $\subterm$, which is defined as follows: $M \subterm N$ iff $N|_p = M$ for some $p \in Pos(N)$.

\section{Flat Termination}

We introduce a definition of program termination inspired by the dependency pair framework which we prove to be a slight generalisation of \emph{size-change termination} \cite{lee2001sizechange,thiemann2005sizechange}.

\begin{definition}[Flat termination]
    Consider some TRS $P \in T(\Sigma_{def} \cup \Sigma{con}, V)^2$. We define its \emph{flattening} $P^\flat$ as follows:
    \begin{equation*}
        P^\flat := \{ f(s_1, \cdots, s_n) \approx g(t_1^\flat, \cdots, t_n^\flat) ~|~ f(s_1, \cdots, s_n) \approx C[g(t_1, \cdots, t_n)] \in P \}
    \end{equation*}
    $P$ is said to be \emph{flat-terminating} iff $P^\flat$ is innermost terminating.
\end{definition}


\renewcommand\em{\it}
\printbibliography[title={References}]

\end{document}