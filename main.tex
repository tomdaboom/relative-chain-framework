\documentclass{article}
\usepackage{graphicx} % Required for inserting images

\usepackage[a4paper, total={6.5in, 10.5in}]{geometry}
\usepackage{hyperref}

\usepackage{amsmath}
\usepackage{amssymb}
\usepackage{amsthm}

\newtheorem{theorem}{Theorem}[section]
\newtheorem{lemma}[theorem]{Lemma}

%\theoremstyle{definition}
\newtheorem{definition}[theorem]{Definition}
\newtheorem{example}[theorem]{Example}
\newtheorem{assumption}[theorem]{Assumption}
\newtheorem{corollary}[theorem]{Corollary}


\AtBeginEnvironment{definition}{\renewcommand\em{\bfseries\textit}}
\renewcommand\em{\bfseries}

\newcommand{\agap}{\hspace{0.25em}}
\newcommand{\dpair}[2]{\langle #1, #2 \rangle}

\usepackage[backend=biber]{biblatex}
\addbibresource{./ref.bib}

\title{The Relative Chain Framework for Modular Termination}
\author{\href{mailto:oi24939@bristol.ac.uk}{\texttt{Tom.Divers@bristol.ac.uk}}, \href{mailto:eddie.jones@bristol.ac.uk}{\texttt{Eddie.Jones@bristol.ac.uk}}}
\date{}
 
\begin{document}

\maketitle

\section{Introduction}


\section{Preliminaries}

\subsection{Term Rewrite Systems}

We consider the setting of term rewriting systems over a \emph{signature} $T(\Sigma, V)$. A signature consists of a finite set of \emph{function symbols} $\Sigma$ and an infinite set of \emph{variables} $V$. $\Sigma$ is presumed to be partitioned into a set of \emph{defined functions} $\Sigma_{def}$ and \emph{constructors} $\Sigma_{con}$. Each $f \in \Sigma$ has an associated \emph{arity}, and $\Sigma^{(i)}$ denotes all the function symbols in $\Sigma$ with arity $i \in \mathbb{N}$.

$T(\Sigma, V)$ is defined inductively as follows: \begin{enumerate}
    \item $\Sigma^{(0)} \subseteq T(\Sigma, V)$ and $V \subseteq T(\Sigma, V)$. 
    \item If $f \in \Sigma^{(n)}$, and $M_i \in T(\Sigma, V)$ for each $i \in [n]$, then $f(M_1, \cdots, M_n) \in T(\Sigma, V)$. 
\end{enumerate}

The recursive structure of $T(\Sigma, V)$ prompts us to think of terms in a TRS as \emph{syntax trees}. We define $Pos(M) \subseteq \mathbb{N}^*$ for each $M \in T(\Sigma, V)$ to be the set of positions in $M$'s syntax tree. The root of the tree has position $\epsilon$; and the $i$th child of a node at position $p$ has position $pi$. $M|_p$ denotes $M$'s subterm located at position $p$, and $M[N]_p$ denotes $M$ with its term at position $p$ replaced by $N \in T(\Sigma, V)$. 

A \emph{(one-hole) context} $C[\cdot] : T(\Sigma, V) \rightarrow T(\Sigma, V)$ is a term with a hole $\circ$ at some position. $C[M]$ denotes the context $C[\cdot]$ with its hole replaced by the term $M$. A \emph{substitution} $\theta : X \rightarrow T(\Sigma, V)$ is a function mapping variables to terms, for which the set $dom(\theta) := \{ x ~|~ \theta(x) \neq x\}$ is finite. $M \theta$ denotes the term $M$ with each variable replaced by its $\theta$-image. 
\\~\\
A reduction system $(R, \rightarrow)$ consists of a set $R$ equipped with a binary relation $\rightarrow ~\subseteq R \times R$. A reduction is \emph{terminating} iff there exist no sequences $x_1x_2 \cdots \in R^\omega$ with $x_1 \rightarrow x_2 \rightarrow \cdots$. A reduction system over $T(\Sigma, V)$ is a \emph{term rewrite syste (TRS)} iff it is closed under contexts and substitutions (i.e. if $M \rightarrow N$, then $C[M\theta] \rightarrow C[N\theta]$ for all contexts $C[\cdot]$ and substitutions $\theta$). 

Here, we will consider TRSs defined by a set of oriented equations $R \subseteq T(\Sigma, V) \times T(\Sigma, V)$. For an equation $M \approx N \in R$, we have that $C[M\theta] \rightarrow_R C[N\theta]$ for all contexts $C[\cdot]$ and substitutions $\theta$. We say that an equation $M \approx N$ is \emph{stable} iff $M$ is headed by a defined function symbol. A TRS is stable iff all of its equations are stable.  


\subsection{The Dependency Pair Framework}

\begin{theorem}[\cite{arts2000dependency}]\label{thm:no_infinite_chains}
    A TRS is terminating iff it does not induce an infinite chain.
\end{theorem} 

\section{Relative Chains}

\begin{definition}
    Consider two TRSs $R$ and $S$, and assume that $R$ comes equipped with dependency pairs $\dpair{\cdot}{\cdot}_R$. A sequence of these pairs $\dpair{s_1}{t_1}_R ~ \dpair{s_2}{t_2}_R \cdots$ forms an $R/S$-chain iff there exists some sequence of substitutions $\theta_1 \theta_2 \cdots$ satisfying $t_i \theta_i \rightarrow^*_S s_{i+1}\theta_{i+1}$ for each $i$. 
\end{definition}

\begin{definition}
    Consider some TRS $S := R_1 \cup R_2$, which is the union of two other TRSs. An $S$-chain is \emph{spanning} iff it is of the form $\cdots p_1 p_2 \cdots$ where $p_1 \in DP(R_1)$ and $p_2 \in DP(R_2)$. In this case, we say that this chain spans \emph{from} $R_1$ \emph{to} $R_2$.
\end{definition}

\begin{theorem}
    Consider two TRSs $R_1$ and $R_2$. $S := R_1 \cup R_2$ is terminating if both of the following hold: \begin{enumerate}
        \item There are no infinite $R_1/S$- or $R_2/S$-chains. \label{cond:no_r_chains}
        \item Every spanning $S$-chain only spans from $R_1$ to $R_2$ (or vice versa). \label{cond:no_span_chains}
    \end{enumerate}
\end{theorem}

\begin{proof}
    We proceed by contrapositive. Assume that $S$ is nonterminating. Hence, by Theorem \ref{thm:no_infinite_chains}, $R_1 \cup R_2$ induces an infinite chain. We now demonstrate that any such chain violates one of our conditions. Observe that an infinite chain must repeat some finite sequence $s$ of dependency pairs infinitely many times. This cycle must occur either entirely within $DP(R_1)$ or $DP(R_2)$, or must alternate between $DP(R_1)$ and $DP(R_2)$. More formally, we have two cases: \begin{enumerate}
        \item $s \in DP(R_1)^* \cup DP(R_2)^*$. Hence $R_1$ or $R_2$ induces an infinite chain, which violates condition \eqref{cond:no_r_chains}. 
        
        \item $s = \cdots x_1 x_2 \cdots y_2 y_1 \cdots$, where $x_i, y_i \in DP(R_i)$ for $i \in \{1, 2\}$ (i.e. $s$ spans from $R_1$ to $R_2$ and from $R_2$ to $R_1$). This directly violates condition \eqref{cond:no_span_chains}. 
    \end{enumerate}
    % ~\\
    % ($\Rightarrow$) Again, we proceed by contrapositive. Assume that condition \eqref{cond:no_r_chains} fails. As before, this immediately induces an infinite $S$-chain. Now assume that condition \eqref{cond:no_span_chains} fails.  
\end{proof}

\section{Flat Termination and Stability}

\renewcommand\em{\it}
\printbibliography[title={References}]

\end{document}