\documentclass{article}
\usepackage{graphicx} % Required for inserting images

\usepackage[a4paper, total={7in, 10.5in}]{geometry}
\usepackage{hyperref}

\usepackage{amsmath}
\usepackage{amssymb}
\usepackage{amsthm}
\usepackage{mathtools}

\newtheorem{theorem}{Theorem}%[section]
\newtheorem{lemma}[theorem]{Lemma}
\newtheorem{corollary}[theorem]{Corollary}


\theoremstyle{definition}
\newtheorem{definition}[theorem]{Definition}
\newtheorem{example}[theorem]{Example}
\newtheorem{assumption}[theorem]{Assumption}


\AtBeginEnvironment{definition}{\renewcommand\em{\bfseries\textit}}
\renewcommand\em{\bfseries}

\newcommand{\agap}{\hspace{0.25em}}
\newcommand{\dpair}[2]{\left\langle #1, #2 \right\rangle}
\newcommand{\subterm}{\sqsubset}
\newcommand{\subtermeq}{\sqsubseteq}
\newcommand{\supterm}{\sqsupset}
\newcommand{\suptermeq}{\sqsupseteq}


\usepackage[backend=biber]{biblatex}
\addbibresource{./ref.bib}

\title{The Relative Chain Framework for Modular Termination}
\author{\href{mailto:oi24939@bristol.ac.uk}{\texttt{Tom.Divers@bristol.ac.uk}}, \href{mailto:eddie.jones@bristol.ac.uk}{\texttt{Eddie.Jones@bristol.ac.uk}}}
\date{}
 
\begin{document}

\maketitle

\section{Introduction}

We present a new framework of \emph{relative chains} for proving the termination of the union of two terminating rewrite systems. We demonstrate how our framework can be used to prove the termination of functional programs augmented with equational hypotheses, and \emph{...}

\section{Preliminaries}

In this section, we discuss some mathematical preliminaries concerning term rewriting and reduction relations. The foundations of this field are explored in depth in \cite{baader1998terms}.

\subsection{Term Rewrite Systems}

We consider the setting of term rewriting systems over a \emph{signature} $T(\Sigma, V)$. A signature consists of a finite set of \emph{function symbols} $\Sigma$ and an infinite set of \emph{variables} $V$. $\Sigma$ is presumed to be partitioned into a set of \emph{defined functions} $\Sigma_{def}$ and \emph{constructors} $\Sigma_{con}$. Each $f \in \Sigma$ has an associated natural number called its \emph{arity}, and $\Sigma^{(i)}$ denotes all the function symbols in $\Sigma$ with arity $i$.

$T(\Sigma, V)$ is defined inductively as follows: \begin{enumerate}
    \item $\Sigma^{(0)} \subseteq T(\Sigma, V)$ and $V \subseteq T(\Sigma, V)$. 
    \item If $f \in \Sigma^{(n)}$, and $M_i \in T(\Sigma, V)$ for each $i \in [m], m \leq n$, then $f ~ M_1 ~ \cdots ~ M_m \in T(\Sigma, V)$. 
\end{enumerate}

The recursive structure of $T(\Sigma, V)$ prompts us to think of terms in a TRS as \emph{syntax trees}. We define $Pos(M) \subseteq \mathbb{N}^*$ for each $M \in T(\Sigma, V)$ to be the set of positions in $M$'s syntax tree. The root of the tree has position $\varepsilon$, and the $i$th child of a node at position $p$ has position $pi$. $M|_p$ is the subterm of $M$ rooted at position $p$, and $M[N]_p$ denotes $M$ with its term at position $p$ replaced by $N \in T(\Sigma, V)$. Additionally, we define $Var(M) \subseteq V$ to be the set of variables appearing in a term. $M$ is \emph{closed} iff $Var(M) = \emptyset$. 

A \emph{(one-hole) context} $C[\cdot] : T(\Sigma, V) \longrightarrow T(\Sigma, V)$ is a term with a hole $\square$ at some position. $C[M]$ denotes the context $C[\cdot]$ with its hole replaced by the term $M$. 

A \emph{substitution} $\theta : X \longrightarrow T(\Sigma, V)$ is a function mapping variables to terms, for which the set $\text{dom}(\theta) := \{ x \in V ~|~ \theta(x) \neq x\}$ is finite. Hence, we may write $\theta = [x_1 \mapsto M_1, x_2 \mapsto M_2, \cdots]$ for finitely many $x_i \in V$ and $M_i \in T(\Sigma,V)$. $M \theta$ denotes the term $M$ with each variable replaced by its $\theta$-image, and we call $M\theta$ an \emph{instance} of $M$. For two substitutions $\theta, \sigma$, we say that $\sigma$ is \emph{less general} than $\theta$ (and write $\sigma \leq \theta$) iff there exists some other substitution $\theta'$ such that $\sigma = \theta' \circ \theta$.
\\~\\
A reduction system $(X, \longrightarrow)$ consists of a set $X$ equipped with a binary relation $\longrightarrow ~\subseteq X \times X$. A reduction is \emph{terminating} iff there exist no infinite sequences $x_1x_2 \cdots \in X^\omega$ with $x_1 \longrightarrow x_2 \longrightarrow \cdots$. $x \in X$ is a \emph{redux} of $\longrightarrow$ iff $x \longrightarrow y$ for some $y \in X$. If $x$ is not a redux, we say that it is in \emph{$\longrightarrow$-normal form}. A reduction system over $T(\Sigma, V)$ is a \emph{term rewrite system (TRS)} iff it is closed under contexts and substitutions (i.e. if $M \longrightarrow N$, then $C[M\theta] \longrightarrow C[N\theta]$ for all contexts $C[\cdot]$ and substitutions $\theta$). 

Here, we will consider TRSs defined by a set of equations $R \subseteq T(\Sigma, V) \times T(\Sigma, V)$. Note that our equations are presumed to be \emph{oriented}, meaning that $M \approx N \in R$ does not imply that $N \approx M \in R$. We also assume that $Var(M) \supseteq Var(N)$ for each $M \approx N \in R$. We define $\longrightarrow_R$ such that, for each $M \approx N \in R$, and for all contexts $C[\cdot]$ and substitutions $\theta$, $C[M\theta] \longrightarrow_R C[N\theta]$. 

An equation $M \approx N$ is \emph{stable} iff $M$ is headed by a defined function symbol. A TRS is stable iff all of its equations are stable. A TRS is a \emph{functional program} iff it is stable, and for each rule $f(x_1, \cdots, x_n) \approx N$, each $x_i$ contains no defined function symbols. We say that an equation $M \approx N$ is \emph{$Q$-normal} (where $Q$ is some TRS) iff $N$ is in $Q$-normal form. A TRS is $Q$-normal iff all of its rules are $Q$-normal. 


A TRS $\longrightarrow$ also defines an \emph{innermost reduction system} $\longrightarrow_i$. We define $s \longrightarrow_i t$ if $s \longrightarrow t$ and all proper subterms of $s$ are in normal form. For any context $C[]$, $C[M] \longrightarrow_i C[N]$ if $M \longrightarrow_i N$. A TRS is \emph{innermost terminating} iff its innermost reduction system is terminating. 
\\~\\
A strict \emph{reduction order} $\succ ~ \subseteq T(\Sigma, V)^2$ is an order on terms that is closed under substitutions.  $\succeq$ is defined to be the reflexive closure of $\succ$. A reduction order is also \emph{monotonic} iff it is closed under contexts. $\succ$ is \emph{well-founded} iff every set $K \subseteq T(\Sigma, V)$ has a minimum under $\succ$ (i.e. $\forall K \subseteq T(\Sigma, V), \exists x \in K, \forall y \in K, x \preceq y$).

One particular ordering that will be of use is the subterm ordering over closed constructors $\subterm ~\subseteq T(\Sigma_{con}, \emptyset)^2$, which is defined as follows: $M \subterm N$ iff $N|_p = M$ for some  $p \in Pos(N)$. Trivially, this is a monotonic reduction order. Also note that $\supterm$ does not permit an infinite reduction sequence; if it did, this would imply the existence of infinite terms over constructors, which we can safely exclude in our eager, call-by-value setting.  

\section{Program-Hypothesis Pairs}

\section{Modular Termination via Size-Change Termination}

\begin{definition}[Size-change termination \cite{lee2001sizechange,thiemann2005sizechange}]
    Consider some functional program $P \subseteq T(\Sigma_{con} \cup \Sigma_{def}, V)^2$. For each rule $f ~ s_1 ~ \cdots ~ s_n \approx N$, and for each function symbol $g$ such that $N = C[g ~ t_1 ~ \cdots ~ t_m)]$ for some context $C$, we define a directed bipartite \emph{size-change graph} $G_f^g := (L, R, E)$ with respect to some well-founded ordering on closed constructor terms $\succ ~ \subseteq T(\Sigma_{con}, \emptyset)^2$.

    $G_f^g$ contains a vertex for each argument of $f$ and $g$ (i.e. $L = \{1_f, \cdots, n_f\}$ and $R = \{1_g, \cdots, m_g\}$). Each of $G_f^g$'s directed edges $e \in L \times R \times \{\succ, \succeq\}$ is labelled either with $\succ$ or $\succeq$. For each pair of arguments $s_i$ and $t_j$, if $s_i \succ t_j$, we have $(i_f, j_g, \succ) \in E$. Otherwise, if $s_i \succeq t_j$, we have $(i_f, j_g, \succeq) \in E$.

    We define the concatenation of multiple size-change graphs corresponding to subsequent function calls $G_f^g \circ G_g^h \circ G_h^i \circ \cdots$. .

   $P$ is defined to be \emph{size-change terminating} with respect to $\succ$ iff some infinite path of every concatenation of an infinite sequence of size-change graphs contains infinitely many edges labelled $\succ$.
\end{definition}
Observe that this formalism of size-change termination is stronger than termination. If a program is size-change terminating, then any possible infinite reduction sequence infinitely decrements some constructor term with respect to $\succ$, which violates $\succ$'s well-foundedness.

\begin{theorem}
    Let $P$ be a functional program that is size-change terminating with respect to $\supterm$, and let $H$ be a $P$-normal, closed and terminating hypothesis set disjoint from $P$. Then $P \cup Q$ is innermost terminating. 
\end{theorem}
\begin{proof}
    Define $\longrightarrow ~:=~ \longrightarrow_{P \cup Q}$, and for the sake of contradiction, consider some infinite innermost reduction sequence of $P \cup Q$ starting from some term $s$. Assume without loss of generality that $s ~ \longrightarrow^*_i ~ t$, where all the subterms of $t$ are in normal form. Via our assumptions, we know that $t$ must match the left-hand side of some rule in $H$ or of some rule in $P$.
    \\~\\
    In the first case, there must be some rule $t \approx u \in H$. Via our assumptions of $P$-normality and closure, $u\sigma$ must be in $P$-normal form for all substitutions $\sigma$. Hence, there is no rule in $P$ that matches $u$. It follows that, if $t$ is non-innermost terminating, we can produce an infinite reduction sequence in $H$, which violates $H$'s termination, thus producing a contradiction.
    \\~\\
    In the second case, we must have $t = (f ~ a_1 ~ \cdots ~ a_n)\sigma$ for some rule $f ~ a_1 ~ \cdots ~ a_n \approx u \in P$ and substitution $\sigma$. From our assumptions, we also know that $u\sigma$ is non-innermost terminating. Via our previous case, $u$ cannot match the left-hand side of a hypothesis. This means that $u\sigma = C[g_1 ~ b_1 ~ \cdots ~ b_m]$ for some $g_1 \in \Sigma_{def}$ and context $C[]$, where $g_1 ~ b_1 ~ \cdots ~ b_m$ is non-innermost terminating with all of its subterms in normal form. 
    
    We now have two more cases: the call to $g_1$ is inserted into $u$ by $\sigma$, or it appears directly in $u$. In the first case, we can deduce that the call to $g$ must also appear in $t$ (note that $Var(f ~ a_1 ~ \cdots ~ a_n) \supseteq Var(u)$), which means that $t$ has a subterm that is not in normal form. This violates our assumptions about the structure of $t$; therefore, the call to $g_1$ must appear in $u$.

    Observe that, because it is a minimally nonterminating subterm, we can now apply the same argument to the term $(g_1 ~ b_1 ~ \cdots ~ b_n)\sigma$, yielding an infinite sequence of function calls $g_1 g_2 g_3 \cdots$, where each $g_{i+1}$ appears on the right-hand side of a definition headed by $g_i$. This yields an infinite sequence of size-change graphs $\{G_i^{i+1}\}_{i \in \mathbb{N}}$. Via size-change termination, their concatenation $G_1^2 \circ G_2^3 \circ \cdots$ must contain some path that traverses infinitely many edges labelled $\supterm$. But if such a call sequence were possible, it would induce an infinite sequence of closed constructor terms that decrease under $\supterm$. We have arrived at another contradiction; therefore, no infinite reduction sequence exists in $P \cup H$.
\end{proof}

% \begin{lemma} \label{thm:size_change_max_multi}
%     A program $P$ is size-change terminating iff any maximal multigraph contains an edge labelled $\succ$.
% \end{lemma}




\renewcommand\em{\it}
\printbibliography[title={References}]

\end{document}